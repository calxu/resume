% !TEX TS-program = xelatex

%% LAYOUTS	
\documentclass{resume}

%% FONTS
\usepackage{zh_CN-Adobefonts_external}      % Simplified Chinese Support using external fonts (./fonts/zh_CN-Adobe/)










\begin{document}

% suppress displaying page number
\pagenumbering{gobble}       










\name{徐 \ 彩 \ 旭}

\medskip

\contactInfo {csxucaixu@gmail.com}    {176-1079-3303}    {www.xucaixu.com} 










\section{   \faListAlt  \  基本信息}
   \subsection{  \hspace{ 0.4cm }     \textbf{性\ \ \ \ \ \ \ \ 别}:    \ 男   
                        \hspace{ 3.2cm }     \textbf{学\ \ \ \ \ \ \ \ 历}:   \ 硕士 
                        \hspace{ 3.2cm }     \textbf{政治面貌}: \ 中共党员 }

   \subsection{  \hspace{ 0.4cm }     \textbf{出生日期}:     \ 1993.03
                        \hspace{ 2.25cm }     \textbf{籍\ \ \ \ \ \ \ \ 贯}:     \ 江苏
                        \hspace{ 3.2cm }     \textbf{求职意向}:  \ 算法应用 }

\medskip











\section{   \faUser    \  个人背景}

\datedsubsection{   \textbf{希音}, 深圳南山区 }                                                  {2022.05-至今}
   \faUserMd   \textit{   算法工程师,涉及用户画像/商品生命周期管理/搜推/用增 }

\datedsubsection{   \textbf{滴滴}, 北京海淀区 }                                                  {2020.12-2022.05}
   \faUserMd   \textit{   算法工程师,涉及运筹算法/机器学习 }

\datedsubsection{   \textbf{京东}, 北京大兴区 }                                  {2020.03-2020.10}
   \faUserMd   \textit{   数据挖掘工程师,涉及图谱/指标画像 }

\datedsubsection{   \textbf{百度}, 北京海淀区  }                                 {2018.04-2020.01}
   \faUserMd   \textit{   算法工程师(百度最高奖入围团队),涉及风控模型/策略/分析 }

\datedsubsection{   \textbf{苏州大学}, 江苏苏州  }                                              {2015.09-2018.07}
   \faGraduationCap   \textit{  硕士,计算机技术(个人排名Top 15\%) }

\datedsubsection{   \textbf{东南大学成贤学院}, 江苏南京}                                  {2011.09-2015.07}
   \faGraduationCap   \textit{  学士,计算机科学与技术(个人排名Top 2\%)}

\medskip










\section{   \faFlag    \  个人优势}

\begin{itemize}  [parsep=0.5ex]
   \item  \textbf{个人经验}:7年Top互联网大厂经验,从0到1负责过数十个工业级项目,具备团队管理经验;

   \item   \textbf{落地经验}:熟悉工业界常用的建模方法论和流程,可熟练使用常用算法的语言、工具和框架;
   
   \item  \textbf{业务经验}:具备理解业务痛点问题、对业务问题抽象和建模的能力,具备快速学习和抗压能力;
  
  \item \textbf{合作经验}:能很好地与产品、运营、工程团队合作达成公司业绩目标;
  
  % \item \textbf{技术优势}:了解经典的供应链算法、搜推算法、用增算法原理;
 
\end{itemize}

\medskip










\section{   \faCogs    \  个人技能}

\begin{itemize}  [parsep=0.5ex]

   \item   \textbf{编程语言}:Python = SQL = Shell > Scala > awk > JAVA > C/C++ > C\#;
   \item    \textbf{大数据工具}:Linux、Hadoop、Hive、PySpark;
   \item    \textbf{数据库}:MySQL、PostgreSQL、Hive、KV键值对线上数据库;
   \item    \textbf{机器学习框架}:XGBoost、Liblinear、TensorFlow;
   \item    \textbf{软件工程}:Git、UML、OOA/OOD;
   \item    \textbf{语言水平}:英语听说读写流利,CET-4/CET-6;
   \item    \textbf{其它}:数据结构、Neo4j、LaTeX、MarkDown、Office系列、数据可视化、静态Web设计;
 
\end{itemize}

\medskip










\section{   \faTrophy   \ 工作和学校奖项}


\begin{itemize} [parsep=0.5ex]

%工作期间
\item   \datedline{ 希音优秀员工奖            $\times$ 1    }                                {2023.05-2024.05}
\item   \datedline{ 百度卓越项目奖                  $\times$ 1    }                                         {2018.04-2020.10}

% 硕士期间
\item   \datedline{ 授权专利                              $\times$ 1    }                                         {2015.09-2018.07}
\item   \datedline{ 计算机EI检索会议论文        $\times$  2    }                                         {2015.09-2018.07}
\item   \datedline{ 计算机中文核心期刊论文    $\times$  2    }                                         {2015.09-2018.07}
\item   \datedline{ 计算机软件著作权               $\times$  2    }                                         {2015.09-2018.07}
\item   \datedline{ 全国软件设计大赛决赛三等奖  $\times$ 1    }                                     {2015.09-2018.07}

% 本硕期间
\item   \datedline{ 计算机科学与技术学院奖学金      $\times$  6  }                                  {2011.09-2018.07}

% 本科期间
\item   \datedline{ 计算机能力等级证书                   $\times$    2 }                                  {2011.09-2015.07}
\item   \datedline{ 本科优秀毕业生兼三好学生        $\times$  1 }                                    {2011.09-2015.07}
\item   \datedline{ 本科软件设计大赛三等奖           $\times$  2 }                                     {2011.09-2015.07}

\end{itemize}


\medskip










\section{   \faUsers    \ 项目经历}

%希音
\datedsubsection{  \textbf{ 用户画像 }  }                      {2022.05-2025.05}

\begin{itemize}  [parsep=0.5ex]

  \item   \textbf{  项目说明:  }   {  用户画像体系建设  } 
  \item   \textbf{  主要工具:  }   {  SQL, \ Jupyter, \ LightGBM, \ Linux, \ DNN }
  \item   \textbf{  项目收益:  }   { 画像体系服务于用户增长、搜推推荐,带来100+ ABT实验推全,ABT实验中贡献 1.1亿美元/年的增长; }

\end{itemize}


{   \ \ \ \  串联工程、产品、算法、应用从0到1构建希音用户画像的体系,服务于人群圈选、人群分析、人群扩展、用户特征、用户增长(包括广告投放、营销、激励),搜推召回、重排环节;

\ \ \ \ 主要涉及画像维度如下:
a)基础属性画像:性别、身材、年龄、区域等;
b)用户行为画像:点击行为、访问行为、加车/收藏行为、下单支付行为;
c)用户偏好:品牌偏好、系列偏好、类目偏好、价格偏好、品类偏好;
d)模型画像:用户消费力分层、用户生命周期预测、小B商家预测;
e)质量规范:准确率、覆盖率、运营价值、更新频次、数据监控、PSI监控;
}

\medskip











%希音
\datedsubsection{  \textbf{ 商品生命周期管理 }  }                      {2022.05-2025.05}

\begin{itemize}  [parsep=0.5ex]

  \item   \textbf{  项目说明:  }   {  商品生命周期管理  } 
  \item   \textbf{  主要工具:  }   {  SQL, \ Jupyter, \ LightGBM, \ Linux }
  \item   \textbf{  项目收益:  }   { 参与商品生命周期管理 }

\end{itemize}


{   \ \ \ \  涉及商品全链路生命周期治理,包括品类错放、商品治理、同款比价项目,智能化商品管理;

\ \ \ \ 主要涉及画像维度如下:
a)同款比价;
b)商品治理:类目错放、暴露款成人用品识别、婴童用品识别、攻击性武器、易燃易爆品识别、侵权款识别;
c)商品销量预测;
}

\medskip













%滴滴
\datedsubsection{  \textbf{ 履约成本拆分 }  }                      {2021.05-2021.11}

\begin{itemize}  [parsep=0.5ex]

  \item   \textbf{  项目说明:  }   { 滴滴橙心优选履约成本【个人角色:项目主要负责人】 } 
  \item   \textbf{  主要工具:  }   {  SQL,\ python,\ Linux,\ Shell }
  \item   \textbf{  项目收益:  }   {  滴滴公司表的价值排名 Top 0.26\% (699/264301),服务下游4048个依赖表 }

\end{itemize}


{    \ \ \ \ 
协助业务方看清履约成本的结构,抽象业务需求并对各项成本拆分,构建橙心履约体系基础能力;
该成本拆分项目已经在全公司推广,目前已成为橙心的基础能力,服务9大业务场景。

\ \ \ \ 主要涉及点如下:
a)与业务拉齐目标并抽象出拆分方案,多次核对并拉齐;
b)对仓租、水电、人力、物资、冷链、配、管理人员,7项总成本按一定的业务逻辑进行拆分;
c)配送成本按网约车拼车逻辑算法进行边际成本的拆分;
d)监控指标层面的搭建,包括:区域成本看板、超远团看板、团长商分、低效团看板;

\medskip











\datedsubsection{  \textbf{ 司机配送规划 }  }                      {2020.12-2021.05}

\begin{itemize}  [parsep=0.5ex]

  \item   \textbf{  项目说明:  }   { 司机配送路径规划【个人角色:项目主要负责人】 } 
  \item   \textbf{  主要工具:  }   {  SQL, \ python, \ Linux, \ Jupyter, \ Shell }
  \item   \textbf{  项目收益:  }   {  平均节省距离6km(节省14.8\%),平均节省时间17min(节省20.9\%) }

\end{itemize}


{    \ \ \ \ 
分析司机的配送行为,规划推荐出司机配送顺序;目前已经推广到全国司机。

\ \ \ \ 主要涉及点如下:
a)分析司机的配送行为/配送特点,并找到相关可优化的点;
b)利用基于区块的贪心搜索方式快速上线一版比司机人工配送更好的策略;
c)从产品层面优化基于下一个配送点的位置推荐策略算法,50\%司机反馈良好;
}

\medskip















%京东
\datedsubsection{  \textbf{ 企业征信图谱 }  }                      {2020.03-2020.08}

\begin{itemize}  [parsep=0.5ex]

  \item   \textbf{  项目说明:  }   {  京东数科B端解决方案产品【个人角色:项目主要贡献者】  } 
  \item   \textbf{  主要工具:  }   {  SQL,\ python,\ Linux,\ Neo4j,\ 图存储与计算 }
  \item   \textbf{  项目收益:  }   {  企业图谱产品化对外输出 }

\end{itemize}


{    \ \ \ \ 构建企业图谱基础能力层和应用层,为业务方提供图谱产品输出。

\ \ \ \ 主要涉及技术点如下:
a) 基础能力层:十亿量级顶点与边基础能力构建(图谱1/2度能力、连通子图的多层穿透);
b) 图谱应用层:API接口(投资任职、股权穿透、最终受益人、司法关联);图谱可视化(图存储与设计);企业指标体系(近500个);最短路径计算。 }

\medskip















%百度
\datedsubsection{  \textbf{ 风控挖掘子模型 }  }                      {2019.02-2019.11}

\begin{itemize}  [parsep=0.5ex]

  \item   \textbf{  项目说明:  }   {  百度金融信贷风控-线上风控子模型分【个人角色:项目主要贡献者】  }                
  \item   \textbf{  主要工具:  }   {  XGBoost, \ Liblinear, \ Hive, \ Shell  }
  \item   \textbf{  项目收益:  }   {  上线策略每天拦截百人进件、全网风险监控;  }

\end{itemize}


{    \ \ \ \ 百度搜索Query挖掘强相关的子模型分,供模型和策略使用,主要包括ETL、RFM挖掘方法、效果评估。
主要涉及技术点如下:
a)平台环境搭建(Hadoop/MapReduce/Hive/Spark);
b)数据挖掘:RFM动态窗口滑动子模型、模型的搭建与评估、底层特征自动化ETL与例行;
c)效果评估:模型分相关性分析(梯度/WOE/IV)的评估;
}

\medskip










\datedsubsection{  \textbf{ 黑名单系统 }  }                             {2018.07-2019.08}

\begin{itemize} [parsep=0.5ex]

  \item   \textbf{  项目属性:  }  {  度小满金融信贷风控【个人角色:项目主要贡献者】  }
  \item   \textbf{  主要工具:  }  {  Python, \ Linux, \ Shell, \ MySQL, \ Hadoop, \ MapReduce, \ Hive }
  \item   \textbf{  项目收益:  }  {  线上风控基础设施,每天拦截百人进件; }

\end{itemize}


{  \ \ \ \ 黑名单系统重构与升级,主要有ETL、分布式爬虫、图关联、RFM挖掘子模型、自动化评估。
主要涉及技术点如下:
a) 数据挖掘:分布式爬虫、图关联风险传播、RFM动态窗口滑动特征;
b) 数仓管理:数仓自动化ETL、数据库自动化管理监控;
c) 效果评估:重要指标(Precision/命中数/命中率/Lift)的自动化评估、自动化监控。}

\medskip















%学校
\datedsubsection{  \textbf{基于时空上下文共现的用户关系强度预测}  }                   {2016.01-2017.02}

\begin{itemize} [parsep=0.5ex]

\item   \textbf{ 项目说明: }   {硕士毕业设计【个人角色:个人硕士毕业论文】}
\item   \textbf{ 主要工具: }  { Python3,  \ Linux,  \ C++,  \ MySql,  \ Scikit-Learn,  \ XGBoost }
\item   \textbf{ 项目收益: }  { 2篇一作论文、1个专利、2篇三作论文; }

\end{itemize}


{ \ \ \ \ 利用开源数据(近亿条签到数据)构建分类模型,模型预测任意用户对关系强度。在召回率相同的情况下,精确度比最好的模型提高10\%。
主要涉及技术点如下:
a) 平台搭建(Linux系统、Python3、XGBoost);
b) 特征工程:数据ETL处理与管理,多维度抽取特征;
c) 模型搭建:构建多视角分类器模型、模型调测。
}

\medskip










\datedsubsection{  \textbf{轨迹相似度计算系统}  }                                                             {2016.03-2016.08}

\begin{itemize} [parsep=0.5ex]

  \item   \textbf{ 项目亮点: }  { 国内比赛项目,第五届中国软件杯决赛三等奖 }
  
\end{itemize}

{ \ \ \ \ 用全国轨迹数据搭建计算平台,响应计算相似度的请求,并从数据库中秒级检索出最相似的轨迹。
% 主要涉及技术点如下:
% a)数据的聚类处理与存储(Python3、Linux);
% b)采用BS模式,前端采用JSP+CSS+DIV;后端用Java实现构建基于聚类簇的数据结构(UML),秒级响应并计算相似度。 
}

\medskip










\datedsubsection{  \textbf{公司信誉度评估分类系统}  }              {2015.06-2015.11}

\begin{itemize} [parsep=0.5ex]

  \item   \textbf{ 项目亮点: }  { 机器学习经典问题的应用; }
  
\end{itemize}

{\ \ \ \ 采集公司的异常记录等数据,构造特征构建随机森林分类预测模型,预测公司所属类别。}
% 采集公司的异常记录、基本属性等数据,构造特征并对部分公司进行聚类标记,从而构建随机森林分类预测模型,预测公司所属类别。}
% 主要涉及的技术点如下:
% (a) 数据的采集与清洗(Python3、Linux);
% (b) 数据存储与管理(MySQL);
% (c) 特征工程与模型调测(Scikit-Learn)。
  
\medskip










% \datedsubsection{  \textbf{扫雷游戏设计}  }                       {2014.12-2015.04}

% \begin{itemize} [parsep=0.5ex]

%  \item   \textbf{ 项目亮点: }  { 面向对象思想的应用; }
  
% \end{itemize}



% {\ \ \ \ 扫雷游戏UML建模并Java实现其功能,玩家信息与MySQL数据库对接。
% 主要涉及技术点如下:
% (a) 建立UML用例图和活动图,游戏的图形界面展示;
% (b) 建立UML详细类图、类关联图及交互图,并设计生成雷区的算法;
% (c) 设置测试用例对游戏进行功能、边界、负面三个层面测试。 
% }
  
% \medskip

















\section{   \faBook    \ 学校论文}

\begin{itemize}[parsep=0.5ex]

  \item \textbf{Caixu Xu} and Ruirui Bai. Inferring Social Ties from Multi-view Spatiotemporal Co-occurrence (APWeb-WAIM 2018,第一作者,CCF C类)
  \item \textbf{Caixu Xu}, JianFeng Yan and etc.   Context Co-occurrence Based Relationship Prediction in Spatiotemporal Data (CMSA 2018, 第一作者,EI检索)
  
\end{itemize}

\medskip










\end{document}
