% !TEX TS-program = xelatex

%% LAYOUTS	
\documentclass{resume}

%% FONTS
\usepackage{zh_CN-Adobefonts_external}      % Simplified Chinese Support using external fonts (./fonts/zh_CN-Adobe/)










\begin{document}

% suppress displaying page number
\pagenumbering{gobble}       










\name{徐 \ 彩 \ 旭}

\medskip

\contactInfo {csxucaixu@gmail.com}    {176-1079-3303}    {www.xucaixu.com} 










\section{   \faHome  \  基本信息}
   \subsection{  \hspace{ 0.4cm }     \textbf{性\ \ 别}:    \ 男   
                        \hspace{ 3.2cm }     \textbf{学\ \ 历}:   \ 硕士 
                        \hspace{ 3.2cm }     \textbf{政治面貌}: \ 中共党员 }

   \subsection{  \hspace{ 0.4cm }     \textbf{年\ \ 龄}:     \ 27   
                        \hspace{ 3.2cm }     \textbf{籍\ \ 贯}:     \ 江苏 
                        \hspace{ 3.2cm }     \textbf{求职意向}:  \ 数据挖掘方向 }
  
\medskip















\section{   \faUser    \  个人背景}

\datedsubsection{   \textbf{滴滴}, 北京海淀区  }                                     {2020.12-至今}
   \faUserMd   \textit{   数据挖掘工程师,涉及运筹算法/机器学习}

\datedsubsection{   \textbf{京东-京东数科}, 北京大兴区  }                                  {2020.03-2020.12}
   \faUserMd   \textit{   数据挖掘工程师,涉及企业图谱/企业指标体系 }

\datedsubsection{   \textbf{百度-百度金融}, 北京海淀区  }         {2018.04-2020.03}
   \faUserMd   \textit{   数据挖掘工程师,涉及风控模型/策略/分析 }

\datedsubsection{   \textbf{苏州大学}, 江苏苏州  }                                              {2015.09-2018.07}
   \faGraduationCap   \textit{  硕士,计算机技术(个人排名Top 15\%) }

\datedsubsection{   \textbf{东南大学成贤学院}, 江苏南京}                                  {2011.09-2015.07}
   \faGraduationCap   \textit{  学士,计算机科学与技术(个人排名Top 2\%)}

\medskip










\section{   \faCogs    \  个人优势}

\begin{itemize}  [parsep=0.8ex]
   \item  \textbf{个人经验}:3.5年互联网大厂的项目经验,做过很多数据分析相关的工作,从0到1负责开发过5个工业级项目,重构过2个工业级项目;

   \item   \textbf{落地经验}:熟悉工业界常用的建模方法论和流程,可熟练使用常用数挖方向的语言和框架;
   
   \item    \textbf{合作经验}:能很好地与产品、运营、工程团队合作把项目做大;
   
  \item  \textbf{其它}:具备对业务问题抽象,抓住业务问题本质和建模的能力;具备快速学习、快速迭代和一定的抗压能力;
 
\end{itemize}

\medskip








\section{   \faCogs    \  个人技能}

\begin{itemize}  [parsep=0.5ex]

   \item   \textbf{编程语言}:Python = SQL = Shell > Scala > awk > JAVA > C/C++ > C\#;
   \item    \textbf{大数据工具}:Linux、Hadoop、MapReduce、Hive;了解Spark;
   \item    \textbf{数据库}:MySQL、PostgreSQL、Hive、KV键值对数据库;
   \item    \textbf{机器学习框架}:XGBoost、Liblinear、Word2Vec、LDA;
   \item    \textbf{软件工程}:Git/Gitlab/GitHub、UML、OOA/OOD;
   \item    \textbf{语言水平}:英语读写流利,CET-4/CET-6;
   \item    \textbf{其它}:数据结构、Neo4j、LaTeX、MarkDown、Office系列、数据可视化、静态Web设计;
 
\end{itemize}

\medskip


















\section{   \faUsers    \ 项目经历}






\datedsubsection{  \textbf{ 运筹定价 }  }                      {2021.04-2021.09}

\begin{itemize}  [parsep=0.5ex]

  \item   \textbf{  项目属性:  }   {  配送司机定价  }                
  \item   \textbf{  项目角色:  }   {  项目主要负责人  }
  \item   \textbf{  主要工具:  }   {  SQL, \ Jupyter, \ python,\ Linux }

\end{itemize}


{    \ \ \ \ 协助业务方降本,分析业务问题痛点并进行抽象建模;利用运筹算法解决司机定价问题;
最终该定价方案已经全国推广,最终收益:保证了司机收入的公平性同时实现3\%的降本;

主要涉及技术点如下:
a)对司机进行多因素定价,包括:起步价、团点数、里程、重量、逆向费用;
b)抽象业务问题的优化目标,总定价最小化的同时方差最小;
c)抽象业务问题中的约束,包括:总成本5\%的波动、仓库供应商总成本满足在区间内;
d)监控指标层面的搭建,包括:司机收入分层、司机收入变化、司机留存率;
}

\medskip









\datedsubsection{  \textbf{ 履约成本拆分 }  }                      {2021.05-2021.11}

\begin{itemize}  [parsep=0.5ex]

  \item   \textbf{  项目属性:  }   { 滴滴橙心优选履约成本基础能力 } 
  \item   \textbf{  项目角色:  }   {  项目主要负责人 } 
  \item   \textbf{  主要工具:  }   {  SQL,\ python,\ Linux,\ Shell }

\end{itemize}


{    \ \ \ \ 
协助业务方看清履约成本的结构,抽象业务需求并对各项成本拆分,构建橙心履约体系基础成本能力;
该成本拆分项目已经在全公司推广,最终收益:橙心的基础能力,服务9大业务场景,滴滴公司内部表的价值排行Top 0.26\%(699/264301),服务下游4048个依赖表;

主要涉及点如下:
a)与业务拉齐目标并抽象出拆分方案,多次核对并拉齐;
b)对仓租、水电、人力、物资、冷链、配、管理人员,7项总成本按一定的业务逻辑进行拆分;
c)配送成本按网约车拼车逻辑算法进行边际成本的拆分;
d)监控指标层面的搭建,包括:区域成本看板、超远团看板、团长商分、低效团看板;


\medskip















\datedsubsection{  \textbf{ 司机配送规划 }  }                      {2020.12-2021.05}

\begin{itemize}  [parsep=0.5ex]

  \item   \textbf{  项目属性:  }   { 司机配送路径规划 } 
  \item   \textbf{  项目角色:  }   {  项目主要负责人 } 
  \item   \textbf{  主要工具:  }   {  SQL, \ python, \ Linux, \ Jupyter, \ Shell }

\end{itemize}


{    \ \ \ \ 
分析司机的配送行为,规划推荐出司机配送顺序;
最终收益:目前已经推广到全国司机,平均节省距离6km(节省14.8\%),平均节省时间 17min(节省20.9\%);


主要涉及点如下:
a)分析司机的配送行为/配送特点,并找到相关可优化的点;
b)利用基于区块的贪心搜索方式快速上线一版比司机人工好的策略;
c)从产品层面优化基于下一个配送点的位置推荐策略算法,50\%司机反馈良好;


\medskip













\datedsubsection{  \textbf{ 企业征信图谱 }  }                      {2020.03-2020.08}

\begin{itemize}  [parsep=0.5ex]

  \item   \textbf{  项目属性:  }   {  京东数科B端解决方案产品-企业图谱  }                
  \item   \textbf{  项目角色:  }   {  项目主要贡献者  }
  \item   \textbf{  主要工具:  }   {  SQL,\ python,\ Linux,\ Neo4j,\ 图存储与计算 }

\end{itemize}


{    \ \ \ \ 构建企业图谱基础能力层和应用层,为业务方提供图谱产品输出。
主要涉及技术点如下:
(a) 基础能力层:十亿量级顶点与边基础能力构建(图谱1/2度能力、连通子图的多层穿透);
(b) 图谱应用层:API接口(投资任职、股权穿透、最终受益人、司法关联);图谱可视化(图存储与设计);企业指标体系(近500个);最短路径计算。 }

\medskip










\datedsubsection{  \textbf{ 风控挖掘子模型 }  }                      {2019.02-2019.11}

\begin{itemize}  [parsep=0.5ex]

  \item   \textbf{  项目属性:  }   {  度小满金融信贷风控-线上风控子模型分  }                
  \item   \textbf{  项目角色:  }   {  项目主要贡献者  }
  \item   \textbf{  主要工具:  }   {  XGBoost, \ Liblinear  }

\end{itemize}


{    \ \ \ \ 挖掘强相关的子模型分供模型和策略使用,主要有ETL、RFM挖掘方法、效果评估。
主要涉及技术点如下:
(a)平台环境搭建(Hadoop/MapReduce/Hive/Spark);
(b)数据挖掘:RFM动态窗口滑动子模型、底层特征自动化ETL与例行;
(c)效果评估:单变量相关性分析(梯度/WOE/IV)的评估。}

\medskip










\datedsubsection{  \textbf{ 黑名单系统 }  }                             {2018.07-2019.08}

\begin{itemize} [parsep=0.5ex]

\item   \textbf{  项目属性:  }  {  度小满金融信贷风控-线上风控基础设施  }
\item   \textbf{  项目角色:  }  {  项目主要贡献者  }
\item   \textbf{  主要工具:  }  {  Python, \ Linux, \ Shell, \ MySQL, \ Hadoop, \ MapReduce, \ Hive, \ XGBoost, \ Liblinear  }

\end{itemize}


{  \ \ \ \ 黑名单系统重构与升级,主要有ETL、分布式爬虫、图关联、RFM挖掘子模型、自动化评估。
主要涉及技术点如下:
(a) 数据挖掘:分布式爬虫、图关联风险传播、RFM动态窗口滑动特征;
(b) 数仓管理:数仓自动化ETL、数据库自动化管理监控;
(c) 效果评估:重要指标(Precision/命中数/命中率/Lift)的自动化评估、自动化监控。}

\medskip










\datedsubsection{  \textbf{基于时空上下文共现的用户关系强度预测}  }                   {2016.01-2017.02}

\begin{itemize} [parsep=0.3ex]

\item   \textbf{ 项目属性: }   {硕士毕业设计}
\item   \textbf{ 项目角色: }   {独立研发}    
\item   \textbf{ 主要工具: }  { Python3,  \ Linux,  \ C++,  \ MySql,  \ Scikit-Learn,  \ XGBoost }

\end{itemize}


{ \ \ \ \ 利用开源数据(近亿条签到数据)构建分类模型,模型预测任意用户对关系强度。在召回率相同的情况下,精确度比最好的模型提高10\%。
主要涉及技术点如下:
(a) 平台搭建(Linux系统、Python3、XGBoost);
(b) 数据ETL处理与管理,多维度抽取特征、构建多视角分类器模型、模型调测。}

\medskip










\datedsubsection{  \textbf{轨迹相似度计算系统(第五届中国软件杯决赛三等奖)}  }                                 {2016.03-2016.08}

{ \ \ \ \ 利用全国轨迹数据搭建计算平台,快速响应计算相似度的请求,并从数据库中秒级检索出最相似的轨迹。
主要涉及技术点如下:
(a) 数据的聚类处理与存储(Python3、Linux);
(b) 采用BS模式,前端采用JSP+CSS+DIV;后端用Java实现构建基于聚类簇的数据结构(UML),秒级响应并计算相似度。 }

\medskip










\datedsubsection{  \textbf{公司信誉度评估分类系统}  }              {2015.06-2015.11}

{\ \ \ \ 到各大信息公开网站采集公司的异常记录、基本属性等数据,构造特征并对部分公司进行聚类标记,从而构建随机森林分类预测模型,预测公司所属类别。
主要涉及的技术点如下:
(a) 数据的采集与清洗(Python3、Linux);
(b) 数据存储与管理(MySQL);
(c) 特征工程与模型调测(Scikit-Learn)。
  
\medskip










\datedsubsection{  \textbf{扫雷游戏设计}  }                       {2014.12-2015.04}

{\ \ \ \ 扫雷游戏UML建模并Java实现其功能,玩家信息与MySQL数据库对接。
主要涉及技术点如下:
(a) 建立UML用例图和活动图,游戏的图形界面展示;
(b) 建立UML详细类图、类关联图及交互图,并设计生成雷区的算法;
(c) 设置测试用例对游戏进行功能、边界、负面三个层面测试。 }
  
\medskip

















\section{   \faTrophy   \ 学校奖项}


\begin{itemize} [parsep=0.5ex]

% 硕士期间
\item   \datedline{ 专利                                     $\times$    1    }                                      {2015.09-2018.07}
\item   \datedline{ 计算机EI检索会议论文        $\times$  2    }                                         {2015.09-2018.07}
\item   \datedline{ 计算机中文核心期刊论文    $\times$  2    }                                         {2015.09-2018.07}
\item   \datedline{ 计算机软件著作权               $\times$  2    }                                         {2015.09-2018.07}
\item   \datedline{ 全国软件设计大赛决赛三等奖  $\times$ 1    }                                    {2015.09-2018.07}

% 本硕期间
\item   \datedline{ 计算机科学与技术学院奖学金      $\times$  6  }                                  {2011.09-2018.07}

% 本科期间
\item   \datedline{ 计算机能力等级证书                   $\times$    2 }                                  {2011.09-2015.07}
\item   \datedline{ 本科优秀毕业生兼三好学生        $\times$  1 }                                    {2011.09-2015.07}
\item   \datedline{ 本科软件设计大赛三等奖           $\times$  2 }                                     {2011.09-2015.07}

\end{itemize}


\medskip










\section{   \faBook    \ 学校论文}

\begin{itemize}[parsep=0.5ex]

  \item \textbf{Caixu Xu} and Ruirui Bai. Inferring Social Ties from Multi-view Spatiotemporal Co-occurrence (APWeb-WAIM 2018,第一作者,CCF C类)
  \item \textbf{Caixu Xu}, JianFeng Yan and etc.   Context Co-occurrence Based Relationship Prediction in Spatiotemporal Data (CMSA 2018, 第一作者,EI检索)
  
\end{itemize}

\medskip










\end{document}
