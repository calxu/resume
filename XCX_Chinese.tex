% !TEX TS-program = xelatex
% !TEX encoding = UTF-8 Unicode
% !Mode:: "TeX:UTF-8"

\documentclass{resume}
\usepackage{zh_CN-Adobefonts_external} % Simplified Chinese Support using external fonts (./fonts/zh_CN-Adobe/)
%\usepackage{zh_CN-Adobefonts_internal} % Simplified Chinese Support using system fonts
\usepackage{linespacing_fix} % disable extra space before next section
\usepackage{cite}

\begin{document}
\pagenumbering{gobble} % suppress displaying page number

\name{徐彩旭}

% {E-mail}{mobilephone}{homepage}
% be careful of _ in emaill address
\contactInfo{csxucaixu@gmail.com}{(+86)\ 188-9697-0393}{}
% {E-mail}{mobilephone}
% keep the last empty braces!
% \contactInfo{xxx@yuanbin.me}{(+86) 131-221-87xxx}{}

\section{\faLightbulbO\  基本信息}

  \subsection{\hspace{0.2cm} \textbf{性\ \ \ 别}:\ 男 \hspace{6.97cm}      \textbf{出生年月}:\ 1993.03.12}
  \subsection{\hspace{0.2cm} \textbf{籍\ \ \ 贯}:\ 江苏盐城 \hspace{5.7cm} \textbf{政治面貌}:\ 中共党员}
  \subsection{\hspace{0.2cm} \textbf{学\ \ \ 历}:\ 硕士在读 \hspace{5.7cm} \textbf{研究方向}:\ 大数据研究与应用}
\medskip


\section{\faGraduationCap\  教育背景}
\datedsubsection{\textbf{苏州大学}, 江苏,苏州}{2015-2018}
\textit{在读硕士}\ 计算机技术 \par
\textit{专业排名前15\%}

\datedsubsection{\textbf{东南大学成贤学院}, 江苏,南京}{2011-2015}
\textit{学士}\ 计算机科学与技术 \par
\textit{专业排名前3\%}
\medskip

\section{\faUsers\ 实习/项目经历}

\datedsubsection{\textbf{百度金融黑名单系统}}{2017.06-至今}
\role{项目属性:}{暑假实习项目(百度金融部)}
\role{项目角色:}{项目贡献者}
\role{主要工具:}{Python2.7,\ Linux,\ MySql,\ Hadoop,\ MapReduce,\ bs4}
{\ \ \ \ 利用Hadoop、MapReduce统计百度内部与金融相关的数据;爬取外部失信数据;内部数据结合外部数据提高黑名单系
统的Precision、Recall评估指标。}
\begin{itemize}
  \item 数据的统计;
  \item 数据采集、数据处理及数据管理;
  \item 黑名单系统Precision和Recall的评估;
\end{itemize}
\medskip


\datedsubsection{\textbf{基于位置社交网络的用户关系强度预测}}{2016.01-2017.02}
\role{项目属性:}{机器学习项目}
\role{项目角色:}{独立研发完成}
\role{主要工具:}{Python3,\ Linux,\ C++,\ MySql,\ Scikit-Learn,\ XGBoost}
{\ \ \ \ 利用斯坦福大学开源SNAP数据(近亿条签到数据,几百万用户)构建分类模型,模型预测任意用户对关系强度。
在召回率相同的情况下,精确度比最好的模型提高10\%。}
\begin{itemize}
  \item 平台搭建(Linux系统+MySql+Python3)
  \item 数据处理与管理
  \item 代码编写(多维度抽取特征、构建多视角分类器模型、模型调测)
\end{itemize}
\medskip


\datedsubsection{\textbf{轨迹相似度计算系统}}{2016.03-2016.08}
\role{项目属性:}{比赛项目(中国软件杯)}
\role{项目角色:}{项目主要负责人}
\role{主要工具:}{Java,\ Python3,\ JS,\ UML,\ Linux}
{\ \ \ \ 利用全国近三千万轨迹数据搭建相似度计算平台,平台可以快速响应请求:计算任意两条轨迹的相似度;任意画出一
条轨迹,从数据库中快速检索出与当前轨迹最相似的轨迹,并以地图显示的形式反馈给用户。}
\begin{itemize}
  \item 数据的聚类处理与存储;
  \item 该项目采用BS模式,前台采用JSP+CSS+DIV;
  \item 后台Java实现,构建基于聚类簇的数据结构(以响应快速读取操作);定义相似度并设计高效算法;
\end{itemize}
\medskip


\datedsubsection{\textbf{公司信誉度评估分类系统}}{2015.06-2015.11}
\role{项目属性:}{实习项目(苏州汇誉通数据科技有限公司)}
\role{项目角色:}{项目贡献者}
\role{主要工具:}{Python3,\ Linux,\ Scikit-Learn}
{\ \ \ \ 到各大信息公开网站采集公司的异常记录、公司基本属性等数据;根据数据构造特征;对部分公司进行聚类标记,构建
随机森林回归预测模型,模型输出公司所属类别。}
\begin{itemize}
  \item 数据的采集与清洗;
  \item 数据存储与管理;
  \item 特征工程与模型调测;
\end{itemize}
\medskip


%\datedsubsection{\textbf{扫雷游戏设计}}{2014.12-2015.04}
%\role{项目属性:}{课程毕业设计}
%\role{项目角色:}{独立设计开发}
%\role{主要工具:}{Java,\ UML,\ Swing包,\ MySql}
%{\ \ \ \ 分析扫雷游戏需求并用UML建模(包含初级、中级和高级三个级别),记录进行游戏所花费的时间,完成游
%戏则记录最快玩家所花费的时间,并将这些信息写入数据库。}
%\begin{itemize}
%  \item 利用UML建立用例图和活动图;游戏的图形展示界面利用swing包搭建;
%  \item 利用UML建立详细类图、类关联图及交互图;并设计递归生成雷区的算法;
%  \item 设置完整的测试用例对游戏进行功能、边界、负面三个层面的测试;
%\end{itemize}
%\medskip


%\datedsubsection{\textbf{居民信息管理系统}}{2013.06-2013.09}
%\role{项目属性:}{暑假实习项目(盐城思源网络科技有限公司)}
%\role{项目角色:}{项目贡献者}
%\role{主要工具:}{C\#,\ ASP.Net,\ Html,\ CSS,\ JS}
%{\ \ \ \ 对公司原始的客户信息系统进行升级,本人负责整改与添加新功能,整改用户的UI 界面,整改信息筛选模
%块(改写Sql语句),添加地图显示小区位置模块。}
%\begin{itemize}
%  \item 整体采用ASP.Net框架,前台使用html、CSS、JS技术;
%  \item 后台使用的是C\#语言,数据库使用MySql数据库;
%  \item 改版后的客户信息管理系统交互性极大提高,界面更加简洁大方;
%\end{itemize}
%\medskip

% Reference Test
%\datedsubsection{\textbf{Paper Title\cite{zaharia2012resilient}}}{May. 2015}
%An xxx optimized for xxx\cite{verma2015large}
%\begin{itemize}
%  \item main contribution
%\end{itemize}


\section{\faCogs\ IT 技能}
% increase linespacing [parsep=0.5ex]
\begin{itemize}[parsep=0.5ex]
  \item 编程语言:Python、C++、C、Shell、Java;
  \item 数据库:熟悉 MySql、Oracle;
\end{itemize}
\medskip


\section{\faLanguage\ 语言水平}
% increase linespacing [parsep=0.5ex]
\begin{itemize}[parsep=0.5ex]
  \item 中文(普通话):母语
  \item 英语: 听、说良好,读、写流利, CET-4
\end{itemize}
\medskip


\section{\faExpand\ 社会活动}
\datedline{\textit{计算机科学与技术学院学工助理,}{\ \ 东南大学成贤学院}}{2013}
\datedline{\textit{院DV工作室技术部副部长}}{2012}
\datedline{\textit{院科学与技术协会副部长}}{2012}
\medskip


\section{\faStar\ 获奖情况}
\datedline{\textit{基于时空数据的用户社交关系强度预测软件(计算机软件著作权)}}{2016.05}
\datedline{\textit{苏州大学三等奖学金}}{2015-2016}
\datedline{\textit{东南大学成贤学院优秀毕业生}}{2015.06}
\datedline{\textit{东南大学成贤学院三好学生}}{2013-2014}
\datedline{\textit{东南大学成贤学院综合一等奖学金}}{2013-2014}
\datedline{\textit{全国计算机二级C语言能力等级证书}}{2013}
\datedline{\textit{江苏省高等学校计算机二级VFP能力等级证书}}{2013}
\datedline{\textit{院级二等奖学金}}{2012-2013}
\datedline{\textit{院级三等奖学金}}{2012-2013}
\datedline{\textit{院级三等奖学金}}{2011-2012}
\medskip


\section{\faStarO\ 学科竞赛}
\datedline{\textit{DataCastle:交通线路通达时间预测, 26名}}{2017.01}
\datedline{\textit{第五届“中国软件杯”大学生软件设计大赛决赛三等奖}}{2016.08}
\datedline{\textit{东南大学第五届程序设计竞赛二等奖}}{2014.05}
\datedline{\textit{院级网页设计制作大赛一等奖}}{2011.12}
\medskip


%% Reference
%\newpage
%\bibliographystyle{IEEETran}
%\bibliography{mycite}
\end{document}
